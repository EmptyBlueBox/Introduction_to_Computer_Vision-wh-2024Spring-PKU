\section{Instance Segmentation}
	
\subsection{Mask R-CNN}

	在目标检测上再进一步,输出哪个pixel属于哪个segmentation.
	
	两种方法:bottom-up, top-down.目前在2D中前者较好,因为bbox已经做得很好.后者在3D中有用.
	
	Top-Down Approach:Mask R-CNN\footnote{何恺明是如何让这个看起来大家都看不起的工作拿到了ICCV 2017 best paper呢?}
	
	首先,经过RoI pooling之后,分辨率可能下降.但这是大家都知道的.
	
	最重要的在于,何恺明指出,我们不能进行最近邻的吸附,否则会不匹配,这是不可能被学习到的.因此何恺明使用双线性插值进行处理,RoI align.
	
	Ablation Study on RoI Align.AP at 75 提升比AP50还高,这是因为这样的方法对于高精度影响更大.此外,加入align后,bounding box的表现也提升了.额外的信息量.synergy.(what is mask?)
	
\subsection{3D Object Detection and Instance Segmentaiton}

	3D object detection部分过于复杂,只做了解,不再详细记录.
	