\section{3D data}

from sensor or graphics.

第三种sensor:time-of-light.即用光的传播时间计算深度.两种:iToF and dToF.direct or indirect.前者使用脉冲波,后者计算光的相位差.

dToF精度高,但对器件要求高,需要使用SPAD(单光子雪崩二极管),无法做得很密集,造价高.iToF精度低(误差与距离成正比),但造价更低.

multiple 3D representation:

multiview images.从各个视角获得的多张图片.它包含3D信息,indirect,not a true 3D representation.

depth image.只知道深度,不知道相机内参,无法确定两个点的距离.因此被称为2.5D.

Voxels.物体占据了位置则设为1.能被index.但是非常昂贵.$O(n^3)$.没有表面的表示.一旦分辨率低,则无法还原丢失的信息.

Irregular 3D representation. Mesh,point Cloud, Implicit representation.

\subsection{Mesh}

在表面取点,用过三点的平面近似表面.实际上也不限于三角形.下面我们讨论triangle mesh.

\begin{equation}
    \begin{aligned}
        V &= \{v_1, \cdots, v_n\} \in \mathbb R^3
        \\
        E &= \{e_1, \cdots, e_n\} \in V\times V
        \\
        F &= \{f_1, \cdots, f_n\} \in V\times V \times V
    \end{aligned}
\end{equation}


好的mesh:watertight(不漏), manifold(外法向量连续).

\subsection{Point Cloud}

Point cloud是一个$3n$级别的存储方式,当然也可以添加其他分量.它是不规则(irregular)且无序的数据.换言之,其数据的序并不是必要的,
不能提供额外信息,如同set一样.好的算法应该尽可能不去运用序.它是非常轻量级,紧致的,线性级别的存储空间.容易存储,容易理解和生成,容易在其上构建算法.

当然它也不是完美的,比如并不容易通过它判断表面的位置.点云实际上是在表面上采样,那么怎样从一个mesh surface上取样呢?

\textbf{Sampling Strategy:Uniform Sampling.}
计算每个表面的面积,以面积为权独立同分布地选取三角形.在三角形里如何均匀选取呢?仿射变换形成直角三角形,再拼接成矩形.
\footnote{2022笔者内心OS:直接拼接成平行四边形如何?2024笔者注:对三角形均匀采样可以在正方形 (平行四边形)中采样然后对角线截半.}

但是这样取样之后,在取样量不算非常大的时候,经常会出现不太均匀的情形.此时我们可以采用Farthest Point Sampling(FPS).
目标是选取$N$个点使其两两距离求和最大.但这是一个$\mathcal{NP}$hard的问题.我们采取一个近似的贪心算法.
我们先均匀选取10000个点.在这10000个点最远的1024个点成为组合优化问题.但仍然比较困难.我们先确定一个点,
最后依次选取最远的点.我们也只是希望点云看起来比较均匀,因此没必要一定取得数学上的最优解.

除此之外,点云的距离度量也成为一个问题,这也是无序带来的问题之一.
我们希望找到一个permutation invariant的度量:Chamfer distance.
\footnote{在某些文献当中,式 \ref{eq:chamfer distance}有时也会带平方.
但是如果带平方,则三角不等式不成立.}
\begin{equation}
    d_{CD} = \sum_{x \in S_1} \min_{y \in S_2} \norm{x - y}_{2} + \sum_{y \in S_2} \min_{x \in S_1} \norm{x - y}_{2}
    \label{eq:chamfer distance}
\end{equation}

对于每个单项,称为uni chamfer distance.在一个点云是另一个子集的时候有用.

另一个度量是Earth Mover's distance\footnote{在WGAN中使用.2024笔者注.}.与CD不同的是,它要求两个点云数量相同,
且每个点必须找到互不重复的对应.\footnote{Earthmover,中文直译为推土机.
EMD距离用于衡量(在某一特征空间下)两个多维分布之间的dissimilarity,它
的计算基于著名的运输问题.但是精确求解EMD亦非易事,调用的package也多为近似算法.}

\begin{equation}
    d_{E M D}\left(S_{1}, S_{2}\right)=\min _{\phi: S_{1} \rightarrow S_{2}} \sum_{x \in S_{1}} \norm{x-\phi(x)}_{2}
\end{equation}

CD对于取样情况不太敏感,而EMD则比较敏感.比如同样对于Stanford bunny,
如果一个点云多集中在头部,另一个比较均匀,则CD变化不大而EMD变换显著.
由于点云是surface+sampling,因此如果对于取样有要求,应该使用EMD.

\subsection{CD vs EMD}

区别在于 CD 对于采样不敏感,而 EMD 对于采样敏感.
同时EMD需要两个点云点的数量相同.

也就是如果mesh一样但是sample不一样,那么EMD会很大

\subsection{Implicit Shape}

另一种表示3D数据的方法是隐式表达,即SDF(signed distance function).
简单地说,这个函数表示空间中点到物体表面的距离,内部为负,外部为正.
其数学定义为:设$\Omega$为度量空间$X$的子集,$d$为$X$的度量,则SDF定义为

\begin{equation}
    f(x)=\begin{cases}
        -d(x, \partial \Omega) & \text { if } x \in \Omega 
        \\
        d(x, \partial \Omega) & \text { if } x \in \Omega^{c}
    \end{cases}
\end{equation}
其中$d(x, \partial \Omega) \xlongequal{\text{def}} \inf_{y \in \partial \Omega} d(x, y)$.

另外,SDF还满足Eikonel equation,即
\begin{equation}
    \norm{\nabla F} = 1.
\end{equation}
这点不难理解,因为沿与距离最短的点的连线方向的方向导数之模为$1$,而函数本身即表达距离,
不可能只前进一个单位的长度,距离变化超过一单位,因此这也是最大值.

implicit representation的形式非常具有潜力,因为它可以很容易地与神经网络结合.
因此我们下一个主题就是 3D的Deep learning.
