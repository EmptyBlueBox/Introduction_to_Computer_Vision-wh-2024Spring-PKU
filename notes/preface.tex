\section*{前言}
这本笔记是作者于2022年春信息科学技术学院王鹤老师开设的计算机视觉导论课程期间的笔记.王鹤老师在Stanford获得Ph.D学位,课程中也毫不令人意外地带有许多\href{https://cs231n.github.io/}{CS231n: Convolutional Neural Network for Visial Recognition}和\href{https://web.stanford.edu/class/cs231a/course_notes.html}{CS231A: Computer Vision, From 3D Reconstruction to Recognition}等课程的影子.课程从对计算机视觉领域的传统方法的介绍开始,介绍了CNN和诸多深度学习的基本知识,如BatchNorm,Regularization等.随后进入3D视觉部分,详细介绍了Pinhole Camera这一模型以及相机标定,对极几何等相关知识.期中之后转入3D数据,语义分割,物体位姿判定以及RNN和生成模型部分.

笔记主要是对王鹤老师上课内容的记录,部分内容由笔者在课余时间了解后添加,这些内容都给出了参考文献或链接.除此之外,笔者还依惯例添加了几节附录,以补充正文当中一些没有展开的细节,以供参考.

这门课是笔者三年以来在信科上过的水准最高的课程,无论是课程内容,教师讲授水平,作业质量,考试区分度还是答疑,都是笔者体验过的课程中最高水准的一档.若信科未来能有一半专业课能达到本课的水平,则世界一流大学指日可待(.

最后,感谢王鹤老师和张嘉曌,陈嘉毅两位助教.笔者曾多次向张助教询问问题,均得到了细致的回答,在此一并表示感谢.

\rightline{林晓疏}