\section*{前言}

这本笔记是作者于2022年春信息科学技术学院王鹤老师开设的计算机视觉导论课程期间的笔记.王鹤老师在Stanford获得Ph.D学位,课程中也毫不令人意外地带有许多\href{https://cs231n.github.io/}{CS231n: Convolutional Neural Network for Visial Recognition}和\href{https://web.stanford.edu/class/cs231a/course_notes.html}{CS231A: Computer Vision, From 3D Reconstruction to Recognition}等课程的影子.课程从对计算机视觉领域的传统方法的介绍开始,介绍了CNN和诸多深度学习的基本知识,如BatchNorm,Regularization等.随后进入3D视觉部分,详细介绍了Pinhole Camera这一模型以及相机标定,对极几何等相关知识.期中之后转入3D数据,语义分割,物体位姿判定以及RNN和生成模型部分.

笔记主要是对王鹤老师上课内容的记录,部分内容由笔者在课余时间了解后添加,这些内容都给出了参考文献或链接.除此之外,笔者还依惯例添加了几节附录,以补充正文当中一些没有展开的细节,以供参考.

这门课是笔者三年以来在信科上过的水准最高的课程,无论是课程内容,教师讲授水平,作业质量,考试区分度还是答疑,都是笔者体验过的课程中最高水准的一档.若信科未来能有一半专业课能达到本课的水平,则世界一流大学指日可待 (.

最后,感谢王鹤老师和张嘉曌,陈嘉毅两位助教.笔者曾多次向张助教询问问题,均得到了细致的回答,在此一并表示感谢.

\rightline{林晓疏}

\rightline{2022年春}

作为北京大学信息科学技术学院的学生,长期以来饱受糟糕课程质量、糟糕课程作业、糟糕考试难度的折磨.
比如算法设计与分析的等课程的教学质量极低,教考分离,ICS考试一面黑板的考试错误题目订正等等.
在这样的环境下,幸运地遇到了王鹤老师开设的计算机视觉导论课程,内容丰富,作业质量高,考试难度适中,
绝对称得上是精品课程\sout{(与算分这种国家精品课程相区别)}.

王鹤老师将计算机视觉的发展脉络呈现给大家,在这个深度学习时代,
老师并没有完全忽视传统CV的方法,而是挑选了其中具有代表性的工作,这些工作为深度学习时代的CV打下了良好的基础,提供了许多基础工具和数据集的构建方式.
同时老师也更加注重深度学习的基础知识,如 BatchNorm 的特性和与其他 Norm 的区别,许多人仅仅只是会 PyTorch 的积木搭建,但是对于这些基础知识的原理和性质却不甚了解,
导致在实际使用中遇到问题时无法解决,王老师在这方面往往提出 intuitive 的问题,引人深思.

我是在大三下学期选修了这门课程,即使我已经具有了一定的深度学习基础,但是我仍然很享受上课\sout{看回放}的过程,因为对于许多已经了解的知识,王老师会再度给出解释,
总是让我在同一个地方有不同的收获.

我在本学期期中考试之前偶然了解到曾经有学长撰写了一本笔记,但是许多内容已经进行了更新或者删改,因此我联系上林晓疏(笔名)学长,获取了这份笔记的源代码,
并在此基础上进行更新,以飨后人.

该笔记按照讲授先后顺序进行排列,但是章节编排按照知识结构划分,因此章节划分可能与课程进度有所不同.
同时本笔记不能替代课程,只是对这部分知识的总结和思考,建议与课程回放配合食用.

\rightline{Yutong Liang}
\rightline{2024年4月24日}
