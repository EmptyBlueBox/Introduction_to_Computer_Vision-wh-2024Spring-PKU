\section{Transfromation in $\mathbb R^n$\label{transformation}}
在这一小节中我们简单介绍各种变换.

\textbf{等距变换}顾名思义,就是保持距离的变换.在其基本形式当中可以表示为平移加旋转,这也是我们已经接触过的.矩阵表示为
\begin{equation}
	\begin{bmatrix}
		x^\prime
		\\
		y^\prime
		\\
		1
	\end{bmatrix} = 
	\begin{bmatrix}
		\bd R & \bm t
		\\
		\bm 0^\top & 1
	\end{bmatrix}
	\begin{bmatrix}
		x
		\\
		y\\
		1
	\end{bmatrix}
\end{equation}
其中的$\bd R$为旋转矩阵,也是正交矩阵.

\textbf{相似变换}是指形状不变,但可以改变大小和位置的变换.直观地说就是等距变换加上缩放.矩阵表示为
\begin{equation}
	\begin{bmatrix}
		x^\prime
		\\
		y^\prime
		\\
		1
	\end{bmatrix} = 
	\begin{bmatrix}
		\bd S \bd R & \bm t
		\\
		\bm 0^\top & 1
	\end{bmatrix}
	\begin{bmatrix}
		x
		\\
		y\\
		1
	\end{bmatrix}, \quad \bd S = 
	\begin{bmatrix}
		s & 0
		\\ 
		0 & s
	\end{bmatrix}
\end{equation}
其中的$\bd S$表示缩放.它保持边的长度比例和角度不变.理解仿射变换的一种有益方法是将线性变换$\bd A$视作旋转和非均匀缩放的组合.这是因为我们可以对$\bd A$进行SVD:
\begin{equation}
	\bd A = \bd U \bd \Sigma \bd V^\top
\end{equation}

由于$\bd U, \bd V^\top$都是正交矩阵,可以视作旋转,而$\bd \Sigma = \diag \{\sigma_1, \sigma_2\}$则可以视为非均匀缩放.

\textbf{仿射变换}则是一种保持了点,线和平行性的变换.它可以表示为一个线性变化加一次平移.也就是
\begin{equation}
	T(\bm v) = \bd A \bm v + \bm t
\end{equation}

同样在齐次坐标下我们可以写作
\begin{equation}
	\begin{bmatrix}
		x^\prime
		\\
		y^\prime
		\\
		1
	\end{bmatrix} = 
	\begin{bmatrix}
		\bd A & \bm t
		\\
		\bm 0^\top & 1
	\end{bmatrix}
	\begin{bmatrix}
		x
		\\
		y\\
		1
	\end{bmatrix}
\end{equation}
只不过这里的矩阵$\bd A$可以代表任意的线性变换了.

\textbf{射影变换}则只保留了将线映射成线,而不保证平行性.它表示为
\begin{equation}
	\begin{bmatrix}
		x^\prime
		\\
		y^\prime
		\\
		1
	\end{bmatrix} = 
	\begin{bmatrix}
		\bd A & \bm t
		\\
		\bm v & b
	\end{bmatrix}
	\begin{bmatrix}
		x
		\\
		y\\
		1
	\end{bmatrix}
\end{equation}
不难看出它包含了上述所有的变换种类,添加了额外的自由度$\bm v$.注意并不总能够通过缩放矩阵使得$b = 1$,因为$b$可能为$0.$当$b \ne 0$时,射影变换可以做如下分解:
\begin{equation}
	\bd H = \bd H_S \bd H_A \bd S_P = 
		\begin{bmatrix}
			s\bd R & \bm t/v
			\\
			\bm 0^\top & 1
		\end{bmatrix}
		\begin{bmatrix}
			\bd K & \bm 0
			\\
			\bm 0^\top & 1
		\end{bmatrix}
		\begin{bmatrix}
			\bd I & \bm 0
			\\
			\bm v^\top & b
		\end{bmatrix}
		= 
		\begin{bmatrix}
			\bd A & \bm t
			\\
			\bm v & b
		\end{bmatrix}
\end{equation}
其中
\begin{equation}
	\bd A = s\bd R\bd K + \frac{1}{v}\bm t \bm v^\top
\end{equation}
且$\bd K$是满足$\det \bd K = 1$的归一化上三角矩阵.如果限定$s$的符号,它是唯一的,只需做一次QR分解即可得到.

不难看出这个分解是将射影变换分解为相似变换$\bd H_S$,仿射变换$\bd H_A$和一个有约束的透视变换(特殊的射影变换)$\bd H_P$组成.


在射影变换下,四个点的交比仍然保持不变.四个点$P_1, P_2, P_3, P_4$的交比定义为
\begin{equation}
	\text { cross ratio }=\frac{\left\|P_{3}-P_{1}\right\|\left\|P_{4}-P_{2}\right\|}{\left\|P_{3}-P_{2}\right\|\left\|P_{4}-P_{1}\right\|}
\end{equation}
