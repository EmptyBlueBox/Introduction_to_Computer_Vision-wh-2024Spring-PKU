\chapter{Camera Calibration}

我们前面已经知道,从$P_w \to P^\prime$的坐标变换是
\begin{equation}
	\bm P^\prime = \bd M \bm P_w = \bd K
	\begin{bmatrix}
		\bd R & \bd T
	\end{bmatrix} \bm P_w
\end{equation}

其中
\begin{equation}
	\bd{M}=\left(\begin{array}{cc}
		\alpha \boldsymbol{r}_{1}^{T}-\alpha \cot \theta \boldsymbol{r}_{2}^{T}+u_{0} \boldsymbol{r}_{3}^{T} & \alpha t_{x}-\alpha \cot \theta t_{y}+u_{0} t_{z} \\
		\frac{\beta}{\sin \theta} \boldsymbol{r}_{2}^{T}+v_{0} \boldsymbol{r}_{3}^{T} & \frac{\beta}{\sin \theta} t_{y}+v_{0} t_{z} \\
		\boldsymbol{r}_{3}^{T} & t_{z}
	\end{array}\right)
\end{equation}

那么什么是calibration problem呢?如果我们已知世界坐标$P_1, \cdots, P_n$
(形式为$[O_w, i_w, j_w, k_w]$)和对应的图像坐标$p_1,\cdots, p_n$.
我们的目标是通过这些已知数据,获得intrinsic and extrinsic parameters.
这个问题的意义,比如我们希望从图像运动获取实际运动.

问题有11个自由度:5(in)+3(ex-r)+3(ex-t) = 11.需要11个方程,6个点.

对每个$p_i(u_i, v_i)$,我们有
\begin{equation}
	\begin{array}{l}
		\mathrm{u}_{\mathrm{i}}=\frac{\mathbf{m}_{1} \mathrm{P}_{\mathrm{i}}}{\mathbf{m}_{3} \mathrm{P}_{\mathrm{i}}} \rightarrow \mathrm{u}_{\mathrm{i}}\left(\mathbf{m}_{3} \mathrm{P}_{\mathrm{i}}\right)=\mathbf{m}_{1} \mathrm{P}_{\mathrm{i}} \rightarrow \mathrm{u}_{i}\left(\mathbf{m}_{3} \mathrm{P}_{i}\right)-\mathbf{m}_{1} \mathrm{P}_{i}=0 \\
		\mathrm{v}_{\mathrm{i}}=\frac{\mathbf{m}_{2} \mathrm{P}_{\mathrm{i}}}{\mathbf{m}_{3} \mathrm{P}_{\mathrm{i}}} \rightarrow \mathrm{v}_{\mathrm{i}}\left(\mathbf{m}_{3} \mathrm{P}_{\mathrm{i}}\right)=\mathbf{m}_{2} \mathrm{P}_{\mathrm{i}} \rightarrow \mathrm{v}_{i}\left(\mathbf{m}_{3} \mathrm{P}_{i}\right)-\mathbf{m}_{2} \mathrm{P}_{i}=0
	\end{array}
\end{equation}

这样我们可以列出方程组:
\begin{equation}
	\left\{\begin{array}{c}
		u_{1}\left(\mathbf{m}_{3} P_{1}\right)-\mathbf{m}_{1} P_{1}=0 \\
		v_{1}\left(\mathbf{m}_{3} P_{1}\right)-\mathbf{m}_{2} P_{1}=0 \\
		\vdots \\
		u_{i}\left(\mathbf{m}_{3} P_{i}\right)-\mathbf{m}_{1} P_{i}=0 \\
		v_{i}\left(\mathbf{m}_{3} P_{i}\right)-\mathbf{m}_{2} P_{i}=0 \\
		\vdots \\
		u_{n}\left(\mathbf{m}_{3} P_{n}\right)-\mathbf{m}_{1} P_{n}=0 \\
		v_{n}\left(\mathbf{m}_{3} P_{n}\right)-\mathbf{m}_{2} P_{n}=0
	\end{array}\right.
\end{equation}

将$m$展开,获得方程组:
\begin{equation}
	\bd P \bm m = \bm 0
\end{equation}
其中
\begin{equation}
	\mathbf{P}\xlongequal{\operatorname{def}}\left(\begin{array}{ccc}
		\mathbf{P}_{1}^{T} & \mathbf{0}^{T} & -u_{1} \mathbf{P}_{1}^{T} \\
		\mathbf{0}^{T} & \mathbf{P}_{1}^{T} & -v_{1} \mathbf{P}_{1}^{T} \\
		& \vdots & \\
		\mathbf{P}_{\mathrm{n}}^{T} & \mathbf{0}^{T} & -u_{n} \mathbf{P}_{\mathrm{n}}^{T} \\
		\mathbf{0}^{T} & \mathbf{P}_{\mathrm{n}}^{T} & -v_{n} \mathbf{P}_{\mathrm{n}}^{T}
	\end{array}\right)
\end{equation}

以及
\begin{equation}
	\boldsymbol{m}=\left(\begin{array}{c}
		\mathbf{m}_{1}^{\mathrm{T}} \\
		\mathbf{m}_{2}^{\mathrm{T}} \\
		\mathbf{m}_{3}^{\mathrm{T}}
	\end{array}\right)
\end{equation}

但是这个问题过定,且有平凡解.我们先对$\bm m$添加一个constrain:$\norm{m} = 1$.随后用SVD求解如下优化问题:
\begin{equation}
	\begin{split}
		\text{minimize } & \norm{\bd P \bm m}^2
		\\
		\st & \norm{m} = 1
	\end{split}
\end{equation}

进行SVD获得最小的特征向量,即为解.

此时还没有结束,我们需要定出收缩因子.假定
\begin{equation}
	\mathcal{M}=\left(\begin{array}{cc}
		\alpha \boldsymbol{r}_{1}^{T}-\alpha \cot \theta \boldsymbol{r}_{2}^{T}+u_{0} \boldsymbol{r}_{3}^{T} & \alpha t_{x}-\alpha \cot \theta t_{y}+u_{0} t_{z} \\
		\frac{\beta}{\sin \theta} \boldsymbol{r}_{2}^{T}+v_{0} \boldsymbol{r}_{3}^{T} & \frac{\beta}{\sin \theta} t_{y}+v_{0} t_{z} \\
		\boldsymbol{r}_{3}^{T} & t_{z}
	\end{array}\right) \rho
\end{equation}

此时我们发现$M$第三行可以直接定出$\rho$.于是这样我们就可以获得相应参数.

实际上,对这个问题的求解就是进行RQ分解\footnote{关于QR分解,请参见附录 \ref{QR Decomposition}}.我们知道RQ分解就是将一个矩阵$\bd A$分解为:

\begin{equation}
	\bd A =  \bd R \bd Q
\end{equation}

其中$\bd Q, \bd R$分别是正交矩阵和上三角矩阵.当然,得到的正交矩阵可能是瑕旋转矩阵,
即行列式为$-1$的正交矩阵.这种情况我们要检查分解得到的上三角矩阵是否满足相机内参的要求,
比如$\sin \theta > 0$,此时将此列和正交矩阵对应行乘以$-1$即可.

当然并不是所有的6个点都可以.不能在同一平面 (15:50).对于一般的有畸变的相机,有更复杂的非线性处理方式,可能没有解析解.

\section{Some Problems with Camera}

\textbf{问题: 只知道内参或深度信息可以唯一确定一个物体吗?}

只有内参或者深度信息,都不能确定一个物体,必须两个都知道才能得到物体真实数据

只有内参:外参和物体大小同时变化

只有深度信息:物体在平行于相机的那一个平面上的大小可以变化,因为这个平面上的物体是同一个深度

\textbf{问题: 假设我们在知道一个照片以外,还知道每个像素的深度,那么可以找出真实世界中两点的距离吗?}

不可以.因为不知道相机内参,∆u和∆v代表着像素差,无法确定相机参考系下的$\Delta x$,$\Delta y$也就是真实距离差.

\textbf{相机相关计算}

1. Depth back projection: 

$(K, u, v, z) \rightarrow (x,y)$

使用K的定义

2. Camera calibration:                 

$(x,y,z,u,v) \rightarrow K$

如果不知道K.那么就是相机标定

\textbf{问题: 为什么相机标定的时候所有参考点不能在同一个平面上?}

如果所有参考点都在同一个平面上,那么相机的观测将缺乏深度信息,因为所有的标定点都位于一个二维平面内.
这会导致所谓的“退化配置” (degenerate configuration),
在这种配置下,我们不能唯一地确定相机的内外参数,尤其是关于深度和空间位置的信息.

例如:我们无法区分相机距离标定平面远但焦距短,与相机距离标定平面近但焦距长的情况.这两种情况在所有参考点都在同一平面上时可能会产生相似的投影图像.

\textbf{问题: 根据 depth back projection 计算出来的相机坐标系下的$\Delta x$和$\Delta y$是不是 world coordinate 下的距离?}

是的.

因为旋转和平移是保角保距离变换.
