\chapter{Classification}

图片分类是CV领域的核心问题.简单来说,就是给定一张图片,判断其属于何种分类,比如是不是猫或狗等等,这对图片的语义理解非常重要.

但是传统的方法对此类问题难以下手,因为图片通常是由数字的矩阵来描述,而从数字到语义有很大的鸿沟,很难设计某个规则来判定是否属于某类.
比如:对象不同的姿势,不同的摄像机视角,不同的背景信息,不同的光照条件,以及对象被隐藏和类内差异等问题.

对于一个好的图片分类器,应该对上述无关因素不敏感,而这也是data augmentation的意义.比如rotation代表姿势和视角的改变,颜色改变代表光照的变化等.

对于图片分类,我们有下列方法:无参方法有最近邻法,参数方法则可以采用CNN.

\section{Nearest Neighbour Classifier}

所谓最近邻,就是将图片视为高维空间的点,将每个训练数据作为已知点,定义一种图片间距离的度量,选取最近的一个 (或几个)
训练数据的类别作为待判断图片的类别.这是一种非常低效的方法,其完美避开了我们上面说到的应具有的标准,对光照/背景/视角/姿势极为敏感,正确率极低,而且需要存储所有训练集.因此,实际中从不使用此种方法.但是最近邻方法在度量学习当中仍有广泛应用.

\section{Using CNN for image Classification}

选用CNN之后,我们需要面对的问题有两个:选取何种网络结构,以及如何设计损失函数.
如今分类问题的网络范式是Softmax classifier + cross-entropy loss.
\footnote{对二分类问题,也可采用SVM loss.但是扩展的多分类SVM loss在如今已经极少使用了.}

\textbf{\\SoftMax}

SoftMax就是一个${\mathbb R}^k \to {(0, 1)}^k$的映射.

\begin{equation}
	\sigma(z)_i = \frac{\exp{\beta z_i}}{\sum \exp(\beta z_j)}
\end{equation}

一般取$\beta = 1.$当$\beta \to \infty$时,SoftMax变成Argmax.

所以 SoftMax 是 Soft 的 Argmax.

关于loss的设计,如果正确标签是one-hot的,那么我们可以使用负对数概率(NLL)作为损失函数.
但是如果ground truth也是一个概率分布(有时这是人为的),
那么我们就需要对两个概率分布的距离度量给出定义.在信息论领域常用的度量是KL divergence $D(P \parallel Q)$,其定义如下:

\begin{equation}
	D(P \parallel Q) = \sum_{x \in \mathcal X} P(x) \log \frac{P(x)}{Q(x)}.
\end{equation}

这个度量并不满足距离的定义,因为其满足正定性,而不满足对称性和三角不等式.

我们不难看出
\begin{equation}
	D(P \parallel Q) = \underbrace{-\sum_{x \in \mathcal X} P(x)\log Q(x)}_{H(P, Q)} - \underbrace{\xk{-\sum_{x \in \mathcal X} P(x) \log P(x)}}_{H(P)}.
\end{equation}
即KL divergence是相对熵和分布$P$的熵之差.如果$P$是groud truth的分布,那么第二项成为常数,就得到了我们的交叉熵损失函数:
\begin{equation}
	\mathcal L_{CE} = H(P, Q) = -\sum_{x \in \mathcal X} P(x) \log Q(x).
\end{equation}

交叉熵函数在随机初始化时,取值约为$\log (\text{sum of classes})$.它没有上界,有下界$0$.

所以 CrossEntrophyLoss 应该在 $\log{\text{类别数}}$ 开始下降

\section{Cross Entrophy Loss V.S. Accuracy}

1.CEL有可能已经降到了log2,acc仍是0.例子:$\Pr=[0.499,0.501]$,仍然输出错误答案,但是 $loss=\log2$ 很小

2.$acc=100\%$的时候,CEL仍然可能是初始化的 $\log(N)$ , 同理举一个例子:$\Pr=[0.498,0.001,0.001]$

综上所述,两者没有确定关系,训练一定要同时画两个曲线